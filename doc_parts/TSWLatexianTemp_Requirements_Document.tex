\begin{document}

% start: template for headers and footer info that need to be adde in each pade that includes a section
\chead{\textit{IT-University of Copenhagen} \rangle  SSEQ-E2013  \rangle \textbf{Group:} 10 Danish Travel card  \rangle \textbf{ID:} 11 \rangle Responsible: All}
\cfoot{\textbf{Hand-in date:} \today \rangle \textbf{Supervisor:} Marco Nardello \rangle \textbf{Version:} 1 \rangle \textbf{Status: } Done \thepage}
\renewcommand{\headrulewidth}{0.1pt}
\renewcommand{\footrulewidth}{0.1pt}
% ends: headers/footers template

\section*{Software Requirements Specification}

%The Structure of a ours requirement document will look as:
This \textbf{SRS} needs to have some basic structure that will made its use more efficiently when following its guideline, this have to be seeing as a guideline more than an absolute guide, never the less following this document will void major flows and errors in the software while developing the project or afterwards. In this case the basic document architecture consist in the follow:

\begin{itemize}%[leftmargin=1em]
  \renewcommand{\labelitemi}{$\Rightarrow$}
 	\item \textbf{Introduction}: Describe how requirements are handled for waterfall model and the system’s function.
 	
	\item \textbf{Requirements}: Describe the functional system requirements and non-functional requirements.
	
	\item \textbf{Glossary}: Define the technical terms used in this document. 
\end{itemize}

%\subsection*{Introduction}
%Describe how requirements are handled for waterfall model and the system’s function.
%In this part of the document will consist of the description, related theory for the chosen development model which in our case is the \textit{Waterfall}\footnote{http://en.wikipedia.org/wiki/Waterfall_model}, this will also include a description how this model influence the systems development and functionality.

%\subsection*{Requirements}
%Describe the functional system requirements and non-functional requirements.

%\subsection*{Glossary}
%Define the technical terms used in this document. 

\section*{Introduction}
In software development, Requirements phase focuses on defining and capturing the needs and problems that a software application is to address and solve. The requirements phase is the first phase in waterfall development model, setting the stage for the rest of the phases of the software application development. 


The requirements phase aims to come up with Requirement Specification document or documents. The aim is to define the requirements in as clear and as detail manner as possible. Normally in order to capture, collect and gather the requirements; a dedicated team is setup to capture the requirements. 


\section*{Requirement}
In our software requirement document we are going to use the graphical notation and structured specifications, in order to handle and define each requirement. The choice of the graphical notation is based primarily on the fact that it is far more readable for us as developers, the structured language notations use templates to specify system
requirements.


\section*{Software description:}
\textbf{Requirement ID:} Identification number for specific requirement
\textbf{Title:} the title of the specific requirement
\textbf{Description:} Specific description and details on what the requirement functions does, and how to how to proceed when an action takes place
\textbf{Priority:} Define the priority of this requirement for the entire project
\textbf{Risk:} Define the risk if this requirement is not meet.

\section*{Priorities:}
\textbf{Low:} Good to have functionality but not required to make the software work.
\textbf{Medium:} Some functionality that could have to implement the UI/UX while interacting inside the GUI
\textbf{High:} It must have to function

\section*{Risk:}
\textbf{Critical $(C)$:} It must be developed to make the software work
\textbf{Medium $(M)$:} Optional functionality but not essential to make the basic software work
\textbf{Low $(L)$:} Function that will be implemented for as a complement for the software but present no risk for its functionality. 

An adult is a person in the age of $16+$, where a kid is a person in the age $12 - 16$. For kids under the age of $12$ the journey is free of charge, but the person has to be accomplished by an adult or he has to pay the minimum fee.

%------------------------------ 01
\begin{center}
	\def\arraystretch{1.5}%
    \begin{tabular}{ | p{5cm} | p{5cm} |}
    \hline
    	\textbf{Requirement ID} & 01 \\ \hline
		\textbf{Title} & Block Account\\ \hline
		\textbf{Description} & User can block his account that which put a stop to any kind of travel or transactions made with the travel card or mobile application until further notice.\\ \hline
		\textbf{Priority} & High\\ \hline
		\textbf{Risk} & Critical\\
      \hline
    \end{tabular}
\end{center}
%-----------------------------

%------------------------------ 02
\begin{center}
	\def\arraystretch{1.5}%
    \begin{tabular}{ | p{5cm} | p{5cm} |}
    \hline
    	\textbf{Requirement ID} & 02 \\ \hline
		\textbf{Title} & Check Account Balance\\ \hline
		\textbf{Description} & This required the such account has been authenticated, and show the account current balance.\\ \hline
		\textbf{Priority} & High\\ \hline
		\textbf{Risk} & Critical\\
      \hline
    \end{tabular}
\end{center}
%-----------------------------

%------------------------------ 03
\begin{center}
	\def\arraystretch{1.5}%
    \begin{tabular}{ | p{5cm} | p{5cm} |}
    \hline
    	\textbf{Requirement ID} & 03 \\ \hline
		\textbf{Title} & Create Account\\ \hline
		\textbf{Description} & Open an account, which it has to include the username, last name, billing address, telephone, and credit card information. This has to be made true a secure channel (https) with their valid SSL certificate and have a two steps verification to validate users fx: SMS to verify phone.\\ \hline
		\textbf{Priority} & High\\ \hline
		\textbf{Risk} & Critical\\
      \hline
    \end{tabular}
\end{center}
%-----------------------------

%------------------------------ 04
\begin{center}
	\def\arraystretch{1.5}%
    \begin{tabular}{ | p{5cm} | p{5cm} |}
    \hline
    	\textbf{Requirement ID} & 04 \\ \hline
		\textbf{Title} & Check History\\ \hline
		\textbf{Description} & This required to have access to the users travel data, balance history and spending data, to be able to show a complete list or historical spending of the users.\\ \hline
		\textbf{Priority} & High\\ \hline
		\textbf{Risk} & Medium\\
      \hline
    \end{tabular}
\end{center}
%-----------------------------

%------------------------------05
\begin{center}
	\def\arraystretch{1.5}%
    \begin{tabular}{ | p{5cm} | p{5cm} |}
    \hline
    	\textbf{Requirement ID} & 05 \\ \hline
		\textbf{Title} & Check Customer Status\\ \hline
		\textbf{Description} & Make possible for: customer, customer service and controller to check whether the passenger traveling on Rejsekort have checked in.\\ \hline
		\textbf{Priority} & High\\ \hline
		\textbf{Risk} & Critical\\
      \hline
    \end{tabular}
\end{center}
%-----------------------------

%------------------------------06
\begin{center}
	\def\arraystretch{1.5}%
    \begin{tabular}{ | p{5cm} | p{5cm} |}
    \hline
    	\textbf{Requirement ID} & 06 \\ \hline
		\textbf{Title} & SendMail\\ \hline
		\textbf{Description} & Both the customer and service can communicate by mail\\ \hline
		\textbf{Priority} & Medium\\ \hline
		\textbf{Risk} & Medium\\
      \hline
    \end{tabular}
\end{center}
%-----------------------------

%------------------------------ 07
\begin{center}
	\def\arraystretch{1.5}%
    \begin{tabular}{ | p{5cm} | p{5cm} |}
    \hline
    	\textbf{Requirement ID} & 07 \\ \hline
		\textbf{Title} & EraseFine\\ \hline
		\textbf{Description} & The customer service can erase a fine given to a customer.\\ \hline
		\textbf{Priority} & Medium\\ \hline
		\textbf{Risk} & Medium\\
      \hline
    \end{tabular}
\end{center}
%-----------------------------

%------------------------------ 08
\begin{center}
	\def\arraystretch{1.5}%
    \begin{tabular}{ | p{5cm} | p{5cm} |}
    \hline
    	\textbf{Requirement ID} & 08 \\ \hline
		\textbf{Title} & CheckIn\\ \hline
		\textbf{Description} & A passenger traveling with Rejsekort can check-in when he starts his commute and if he wish he can add additional passengers such as adults and kids, bicycles or animals\\ \hline
		\textbf{Priority} & High\\ \hline
		\textbf{Risk} & Critical\\
      \hline
    \end{tabular}
\end{center}
%-----------------------------

%------------------------------ 09
\begin{center}
	\def\arraystretch{1.5}%
    \begin{tabular}{ | p{5cm} | p{5cm} |}
    \hline
    	\textbf{Requirement ID} & 09 \\ \hline
		\textbf{Title} & GiveFine\\ \hline
		\textbf{Description} & The controller can give a passenger a fine.\\ \hline
		\textbf{Priority} & High\\ \hline
		\textbf{Risk} & Critical\\
      \hline
    \end{tabular}
\end{center}
%-----------------------------

%------------------------------ 10
\begin{center}
	\def\arraystretch{1.5}%
    \begin{tabular}{ | p{5cm} | p{5cm} |}
    \hline
    	\textbf{Requirement ID} & 10 \\ \hline
		\textbf{Title} & BlockCardAccount\\ \hline
		\textbf{Description} & A customer or customer service are able to block a specific card\\ \hline
		\textbf{Priority} & High\\ \hline
		\textbf{Risk} & Critical\\
      \hline
    \end{tabular}
\end{center}
%-----------------------------

%------------------------------ 11
\begin{center}
	\def\arraystretch{1.5}%
    \begin{tabular}{ | p{5cm} | p{5cm} |}
    \hline
    	\textbf{Requirement ID} & 11 \\ \hline
		\textbf{Title} & AddBalance\\ \hline
		\textbf{Description} & The customer service or the customer can add balance to his account.\\ \hline
		\textbf{Priority} & High\\ \hline
		\textbf{Risk} & Critical\\
      \hline
    \end{tabular}
\end{center}
%-----------------------------

%------------------------------ 12
\begin{center}
	\def\arraystretch{1.5}%
    \begin{tabular}{ | p{5cm} | p{5cm} |}
    \hline
    	\textbf{Requirement ID} & 12 \\ \hline
		\textbf{Title} & CheckOut\\ \hline
		\textbf{Description} & When a passenger ends his commute he is able to checkout, which also can be done by the customer service in cases it is needed.\\ \hline
		\textbf{Priority} & High\\ \hline
		\textbf{Risk} & Critical\\
      \hline
    \end{tabular}
\end{center}
%-----------------------------

%------------------------------ 13
\begin{center}
	\def\arraystretch{1.5}%
    \begin{tabular}{ | p{5cm} | p{5cm} |}
    \hline
    	\textbf{Requirement ID} & 13 \\ \hline
		\textbf{Title} & AddBicycle\\ \hline
		\textbf{Description} & A passenger traveling with Rejsekort can check-in when he starts his commute and if he wish he can add an bicycle to his journey. It is permitted only with two wheel bicycle.\\ \hline
		\textbf{Priority} & High\\ \hline
		\textbf{Risk} & Medium\\
      \hline
    \end{tabular}
\end{center}
%-----------------------------

%------------------------------ 14
\begin{center}
	\def\arraystretch{1.5}%
    \begin{tabular}{ | p{5cm} | p{5cm} |}
    \hline
    	\textbf{Requirement ID} & 14 \\ \hline
		\textbf{Title} & AddAnimal\\ \hline
		\textbf{Description} & A passenger traveling with Rejsekort can check-in when he starts his commute and if he wish he can add an animal to his journey. For animals meant Small dogs, cats, birds and other small animals. Animals may be included for free, but must be placed in a bag or cage and remain there for the whole journey. For larger dogs mean dogs that can not be carried in a purse, a cage or similar. Each passenger may only include one large dog, and this must be on a leash and under the passenger's control. For dog paid child price.\\ \hline
		\textbf{Priority} & High\\ \hline
		\textbf{Risk} & Medium\\
      \hline
    \end{tabular}
\end{center}
%-----------------------------

%------------------------------ 15
\begin{center}
	\def\arraystretch{1.5}%
    \begin{tabular}{ | p{5cm} | p{5cm} |}
    \hline
    	\textbf{Requirement ID} & 15 \\ \hline
		\textbf{Title} & Add Adults\\ \hline
		\textbf{Description} & A passenger traveling with Rejsekort can check in when he starts his commute and if he wish he can add an adult to his journey.\\ \hline
		\textbf{Priority} & High\\ \hline
		\textbf{Risk} & Medium\\
      \hline
    \end{tabular}
\end{center}
%-----------------------------

%------------------------------ 16
\begin{center}
	\def\arraystretch{1.5}%
    \begin{tabular}{ | p{5cm} | p{5cm} |}
    \hline
    	\textbf{Requirement ID} & 16 \\ \hline
		\textbf{Title} & Add Kids\\ \hline
		\textbf{Description} & A passenger traveling with Rejsekort can check in when he starts his commute and if he wish he can add an animal to his journey.\\ \hline
		\textbf{Priority} & High\\ \hline
		\textbf{Risk} & Medium\\
      \hline
    \end{tabular}
\end{center}
%-----------------------------

%------------------------------ 17
\begin{center}
	\def\arraystretch{1.5}%
    \begin{tabular}{ | p{5cm} | p{5cm} |}
    \hline
    	\textbf{Requirement ID} & 17 \\ \hline
		\textbf{Title} & Withdraw\\ \hline
		\textbf{Description} & Depending on the customers choice, the financial institute withdraws money from the customers bank account when balance is added to the Rejsekort account or when checking out.\\ \hline
		\textbf{Priority} & High\\ \hline
		\textbf{Risk} & Critical\\
      \hline
    \end{tabular}
\end{center}
%-----------------------------

\section*{Glossary}

A functional requirement has the following properties:

%------------------------------ Glossary
\begin{center}
	\def\arraystretch{1.5}%
    \begin{tabular}{ | p{5cm} | p{5cm} |}
    \hline
    	\textbf{Requirement ID} & Uniquely identifies requirement \\ \hline
		\textbf{Title} & Gives the requirement a symbolic name\\ \hline
		\textbf{Description} & The definition of the requirement\\ \hline
		\textbf{Priority} & Defines the order in which requirements should be implemented. Priorities are assigned values from low to high\\ \hline
		\textbf{Risk} & Specifies the risk if this specific requirement is not implemented. It shows how critical the requirements is to the system as a whole. The following risk levels are defined over the impact of not being implemented correctly.
 -Critical (C) it will break the main functionality of the system. The system cannot be used if this requirement is not implemented.
 -High (H) It will impact the main functionality of the system. Some function of the system could be inaccessible, but the system can be generally used.\\
      \hline
    \end{tabular}
\end{center}
%-----------------------------
\end{document}