% start: template for headers and footer info that need to be adde in each pade that includes a section
\chead{\textit{IT-University of Copenhagen} \rangle  SSEQ-E2013  \rangle \textbf{Group:} 10 Danish Travel card  \rangle \textbf{ID:} 101 \rangle Responsible: All}
\cfoot{\textbf{Hand-in date:} \today \rangle \textbf{Supervisor:} Marco Nardello \rangle \textbf{Version:} 1 \rangle \textbf{Status: } Done \thepage}
\renewcommand{\headrulewidth}{0.1pt}
\renewcommand{\footrulewidth}{0.1pt}
% ends: headers/footers template

\section*{Conclusion} 


This course has been very enriching and enjoyable. It has tremendously contributed to our growth as master/bachelor students. Every lecture and document developed in this course has been a challenge, a challenge that we weren’t quite use to handle. Because develop documents was a new experience in our career which forced us to think in a differently. While developing and idea to find the proper solutions to a given problem was also new to us. Which forced us to think in a more abstract way and fully understand the documents, models and methods to develop this ideas.


Before the course we thought this tasks could have been quite easy to achieve, but during the course it became very obvious that we had to put a lot of effort, in order to keep up doing and understanding the objectives.


The course was very well-designed for each activity added to the larger objective of the course. And, we believe the course succeeded in achieving the goal that was set up for us.


Which also meant that we had to argue, for each choice that we made, in order to justify the choice. And not chose based on what we think is \textit{IN}. Like in the choice of the development model, one could argue to chose scrum, cause that is what are popular today (in terms of widespread use in the industry), but ain’t the best choice for our project, because our project description ain’t going to change during the development, but rather fixed. So our choice fell on the waterfall-model as it best fits our project.


The exciting practical component of the course, coupled with strong theoretical foundations, gave us opportunities to explore and experiment with different models in software engineering.

 
We have learned to critically examine the choices we made, and to reflect on how well our models align with the experts in the field.Overall, after finishing this course, I feel that I gained a variety of technical skills that will be helpful in other classes and in our future career. In addition, all the feedback that We got from others will help us focus on the areas that We need to work on. For this reason, We are glad that the final project was to do this e-portfolio because it allowed us to revise and reflect on everything that We have done for this course.
 
After We re-read and revised all of our work, We have to admit that seeing the work that We had completed for this class amazed us. Before the course, it seemed that the workload was a lot, but now We understand how helpful it was to put most of my ideas in written assignments. 