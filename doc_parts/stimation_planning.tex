% start: template for headers and footer info that need to be adde in each pade that includes a section
\chead{\textit{IT-University of Copenhagen} \rangle  SSEQ-E2013  \rangle \textbf{Group:} 10 Danish Travel card  \rangle \textbf{ID:} 42 \rangle Responsible: All}
\cfoot{\textbf{Hand-in date:} \today \rangle \textbf{Supervisor:} Marco Nardello \rangle \textbf{Version:} 1 \rangle \textbf{Status: } Done \thepage}
\renewcommand{\headrulewidth}{0.1pt}
\renewcommand{\footrulewidth}{0.1pt}
% ends: headers/footers template

\section*{Estimation and Planning}


\subsection*{Planning Poker Estimates}

Planning Poker, also call \textbf{Scrum poker}, using this planning model members of the group make estimates by playing numbered cards face down o the table, instead of speaking them loudly, later the cards get reveled and the estimates can be discussed\footnote{http://en.wikipedia.org/wiki/Planning_poker}.

\subsubsection*{Problem Domain Analysis}

\begin{center}
	\def\arraystretch{1.5}%
    \begin{tabular}{ | l |  p{5cm} | p{5cm} |}
    \hline
	\textbf{N$0$} & \textbf{Task} & \textbf{Winning Estimation}\\ \hline
	1 & Explore the user content \begin{enumerate}
		\item Brainstorm on what research is necessary for your case
		\item Implement the research
		\item Document the results in an adequate manner
		\item Refine your documentation
	\end{enumerate}  & Hours \begin{enumerate}
		\item 1
		\item 2
		\item 1
		\item $1/2$
	\end{enumerate} \\ \hline
	2 & Rich pictures \begin{enumerate}
		\item Develop two Rich Pictures
		\item Present the Rich Pictures for the whole group
		\item Use the notes to write a short explanation of the rich pictures
	\end{enumerate} & Hours \begin{enumerate}
		\item 1 $1/2$
		\item We were only “1” group
		\item $1/2$
	\end{enumerate} \\ \hline
	
	3 & Analysis class diagram over problem domain \begin{enumerate}
		\item Develop a class-event table about the problem domain of your case  
		\item Develop a class diagram for the problem domain of your case
		\item Evaluate the class diagram
		\item Describe the classes
	\end{enumerate} & Hours \begin{enumerate}
		\item 1
		\item 1$1/2$
		\item $1/2$
		\item $1/2$
	\end{enumerate}\\ \hline
	
	4 & Project Glossary \begin{enumerate}
	\item Create a glossary
	\item Define the central terms in your analysis
	\end{enumerate} & Hours \begin{enumerate}
		\item $1/2$ 
		\item $1/2$
	\end{enumerate}\\ \hline
	
	5 & Supervisor meeting & $1/2$ hour

	\\
	\hline
    \end{tabular}
\end{center}


Our estimation was done in the group according to the principles of Planning Poker Estimation. We all had squared papers with different numbers within a range, which were representing a certain amount of time. For each task we laid down a random paper on a table facing down. 


This process was repeated for each task until we were satisfied with result of the times given to a task.


\subsection*{Three Point Estimation}
\subsubsection*{Quality in Use}

\begin{left}
	\def\arraystretch{1.5}%
    \begin{tabular}{ | l | p{5cm} | p{2cm} | p{2cm} | p{2cm} | p{2cm} |}
    \hline
	\textbf{N$0$} & \textbf{Task} & \textbf{Best Case} & \textbf{Most likely} & \textbf{Worst Case} & \textbf{Estimation} \\ \hline 
	
	1 & Use case diagrams and scenarios \begin{enumerate}
		\item Brainstorm on the tasks and activities the software is supposed to support
		\item Development of scenarios 
		\item Brainstorm on the actors involved in the tasks
		\item Actor descriptions
		\item Use case diagram
	\end{enumerate} & Hours \begin{enumerate} %bestcase
		\item 1
		\item $1/2$
		\item $1/2$
		\item $1/2$
		\item 2
	\end{enumerate} & Hours \begin{enumerate} % most likely
		\item 1 $1/2$
		\item 1
		\item $1/2$
		\item $1/2$
		\item 2 $1/2$
	\end{enumerate} & Hours \begin{enumerate} % worst case
		\item 2
		\item 1 $1/2$
		\item $1/2$
		\item 1
		\item 3
	\end{enumerate} & Hours \begin{enumerate} % estimation
		\item 1 $1/2$
		\item 1 		
		\item $1/2$
		\item 35 min
		\item 2 $1/2$
	\end{enumerate} \\ \hline 	

	
	2 & Interface design using Mock-ups \begin{enumerate}
		\item Decide on 2 scenarios that you would like to work with in your mock-up workshop
		\item Develop an initial interface design
		\item Prepare 2 mock-up workshops
		\item Implement the 2 mock-up workshops
		\item Document the results
	\end{enumerate} & Hours \begin{enumerate} %bestcase
		\item $1/2$
		\item $1/2$
		\item $1/2$
		\item $1/2$
		\item $1/2$
	\end{enumerate} & Hours \begin{enumerate} % most likely
		\item $1/2$
		\item 1
		\item 1 $1/2$
		\item 2
		\item $1/2$
	\end{enumerate} & Hours \begin{enumerate} % worst case
		\item $1/2$
		\item 1 $1/2$
		\item 2
		\item 2 $1/2$
		\item 1
	\end{enumerate} & Hours \begin{enumerate} % estimation
		\item $1/2$
		\item 1	
		\item 1:25H
		\item 1:50H
		\item 0:35H
	\end{enumerate} \\ \hline 
	
	
	3 & Revising project team protocal \begin{enumerate}
		\item Changes to the Project Team Protocol
		\item Propose and discuss changes in the project team 
	\end{enumerate} & Hours \begin{enumerate} %bestcase
		\item $1/2$
		\item $1/2$
	\end{enumerate} & Hours \begin{enumerate} % most likely
		\item $1/2$
		\item $1/2$
	\end{enumerate} & Hours \begin{enumerate} % worst case
		\item $1/2$
		\item $1/2$
	\end{enumerate} & Hours \begin{enumerate} % estimation
		\item $1/2$
		\item $1/2$		
	\end{enumerate} \\ \hline 
	
	4 & Supervisor meeting & $1/2$ hour & $1/2$ hour & $1/2$ hour & $1/2$ hour
	\\
    \hline
    \end{tabular}
\end{left}

$Estimation =Best Case +4XMost Likely+Worst Case$

In this case we calculated the estimated time for each task based on the three cases; \textit{best case}, \textit{most likely}, and \textit{worst case}. Each case was given a time which we as a group discussed and agreed upon. After all three cases were set, the estimation was calculated based on the estimation formula seen under the table $BestCase + 4 x MostLikely + WorstCase / 6$. The procedure was repeated for all the tasks.

