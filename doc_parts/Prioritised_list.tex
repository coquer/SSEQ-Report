% start: template for headers and footer info that need to be adde in each pade that includes a section
\chead{\textit{IT-University of Copenhagen} \rangle  SSEQ-E2013  \rangle \textbf{Group:} 10 Danish Travel card  \rangle \textbf{ID:} 21 \rangle Responsible: All}
\cfoot{\textbf{Hand-in date:} \today \rangle \textbf{Supervisor:} Marco Nardello \rangle \textbf{Version:} 1 \rangle \textbf{Status: } Done \thepage}
\renewcommand{\headrulewidth}{0.1pt}
\renewcommand{\footrulewidth}{0.1pt}
% ends: headers/footers template

\section*{Prioritised List Of Software Qualities}

\begin{center}
	\def\arraystretch{1.5}%
	\begin{tabular}{| p{4cm} | p{4cm} | p{6cm} |}
	
 		\hline
			Software Qualities & Points & Meaning \\ \hline
   			Safety & 82 & N\slash A \\ \hline
			
			\textbf{Security} & \textbf{590} & Strong code structure in case hacking into system(F.x SQL injection)
			-Unit test
			-Errors report from users \\ \hline
			
			\textbf{Reliability} & \textbf{560} & Computer program’s ability  to perform its intended functions and operations in a system's environment, without experiencing failure (system crash).

			IEEE-Std-729-1991: “Software reliability is defined as the probability of failure-free operation for a specified period of time in a specified environment” 

			-Random testing
			-Redeveloping
			-Testing in different environments \\ \hline
			
			Resilience & 37 & N\slash A \\ \hline
			
			Robustness & 210 & N\slash A \\ \hline
			
			Under-stability & 210 & N\slash A \\ \hline
			
			Testability & 270 & N\slash A \\ \hline
			
			Adaptability & 310 &  N\slash A \\ \hline
			
			Modularity & 85 & N\slash A \\ \hline
			
			Complexity & 120 & N\slash A \\ \hline
			
			Portability & 50 & N\slash A \\ \hline
			
			\textbf{Usability} & \textbf{580} & The quality of user experience across websites, software, products, and environments. It is easy for users to understand and using the system.

			-Usability Design(UI)
			-Feedback from users \\ \hline
			
			Reusability & 82 & N\slash A \\ \hline
			
			Efficiency & 150 & N\slash A \\ \hline
			
			Learnability & 280 & N\slash A \\ 
			
   		 \hline
	\end{tabular}
\end{center}