% start: template for headers and footer info that need to be adde in each pade that includes a section
\chead{\textit{IT-University of Copenhagen} \rangle  SSEQ-E2013  \rangle \textbf{Group:} 10 Danish Travel card  \rangle \textbf{ID:} 53 \rangle Responsible: All}
\cfoot{\textbf{Hand-in date:} \today \rangle \textbf{Supervisor:} Marco Nardello \rangle \textbf{Version:} 1 \rangle \textbf{Status: } Done \thepage}
\renewcommand{\headrulewidth}{0.1pt}
\renewcommand{\footrulewidth}{0.1pt}
% ends: headers/footers template

\section*{Project Glossary}

\begin{center}
	\def\arraystretch{1.5}%
    \begin{tabular}{ | p{5cm} | p{5cm} |}
    \hline 
	\textbf{Term} & \textbf{Definition} \\ \hline
	
	Casual user & A person that  uses the rejsekort 1-4 times a week. \\ \hline
	
	Laggers (user)  & Persons that are slowly to adapt to new technologies. \\ \hline
	
	GPS & Global Positioning System.\\ \hline
	
	Use Case & A list of steps that defines the interaction between the user and the system. \\ \hline
	
	Roles and Responsibilities  & A group members role and the responsibilities in the project for the specified date. \\ \hline
	
	Software Qualities & The qualities that are important and agreed upon in the group, which is the corner-stone of the development model. \\ \hline
	
	Work Breakdown Structure & A way to structure the complete project work. \\ \hline
	
	Risk Analysis & Analyze the risks in the project. \\ \hline
	
	Use Context & The context in which the software is deployed. \\ \hline
	
	Rich Pictures & Represent relevant aspects of the problem and application domain as rich pictures. \\ \hline
	
	Interface design  & The design of the interface that the user sees. \\ \hline
	
	Mock-Up & Is a scale or full-size model of a design or device, used for teaching, demonstration, design evaluation and promotion. \\

    \hline
    \end{tabular}
\end{center}