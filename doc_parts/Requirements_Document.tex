\section*{Software Requirements Specification}

The Structure of a ours requirement document will look as:

\subsection*{Introduction}
Describe how requirements are handled for waterfall model and the system’s function.

\subsection*{Requirements}
Describe the functional system requirements and non-functional requirements.

\subsection*{Glossary}
Define the technical terms used in this document. 

\section*{Introduction}
In software development, Requirements phase focuses on defining and capturing the needs and problems that a software application is to address and solve. The requirements phase is the first phase in waterfall development model, setting the stage for the rest of the phases of the software application development. 


The requirements phase aims to come up with Requirement Specification document or documents. The aim is to define the requirements in as clear and as detail manner as possible. Normally in order to capture, collect and gather the requirements; a dedicated team is setup to capture the requirements. 


\section*{Requirement}
In our software requirement document we are going to use the graphical notation and structured specifications, in order to handle and define each requirement. The choice of the graphical notation is based primarily on the fact that it is far more readable for us as developers, the structured language notations use templates to specify system
requirements.


\section*{Software description:}
\textbf{Requirement ID:} Identification number for specific requirement
\textbf{Title:} the title of the specific requirement
\textbf{Description:} Specific description and details on what the requirement functions does, and how to how to proceed when an action takes place
\textbf{Priority:} Define the priority of this requirement for the entire project
\textbf{Risk:} Define the risk if this requirement is not meet.

\section*{Priorities:}
\textbf{Low:} Good to have functionality but not required to make the software work.
\textbf{Medium:} Some functionality that could have to implement the UI/UX while interacting inside the GUI
\textbf{High:} It must have to function

\section*{Risk:}
\textbf{Critical (C):} It must be developed to make the software work
\textbf{Medium (M):} Optional functionality but not essential to make the basic software work
\textbf{Low (L):} Function that will be implemented for as a complement for the software but present no risk for its functionality. 

An adult is a person in the age of 16+, where a kid is a person in the age 12 - 16. For kids under the age of 12 the journey is free of charge, but the person has to be accomplished by an adult or he has to pay the minimum fee.

\begin{center}
	\def\arraystretch{1.5}%
    \begin{tabular}{ | p{5cm} | p{5cm} |}
    \hline
    	\textbf{Requirement ID} & 01 \\ \hline
		\textbf{Title} & Block Account\\ \hline
		\textbf{Description} & User can block his account that which put a stop to any kind of travel or transactions made with the travel card or mobile application until further notice.\\ \hline
		\textbf{Priority} & High\\ \hline
		\textbf{Risk} & Critical\\
      \hline
    \end{tabular}
\end{center}

